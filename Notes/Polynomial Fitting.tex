\documentclass[letterpaper,12pt]{article}

	\usepackage{threeparttable}
	\usepackage[format=hang,font=normalsize,labelfont=bf]{caption}
	\usepackage{amsmath}
	\usepackage{array}
	\usepackage{delarray}
	\usepackage{amssymb}
	\usepackage{amsthm}
	\usepackage{natbib}
	\usepackage{setspace}
	\usepackage{float,color}
	\usepackage[pdftex]{graphicx}
	\usepackage{hyperref}
	\usepackage{multirow}
	\usepackage{enumerate}
	\hypersetup{colorlinks,linkcolor=red,urlcolor=blue,citecolor=red}
	\theoremstyle{definition}
	\newtheorem{theorem}{Theorem}
	\newtheorem{acknowledgement}[theorem]{Acknowledgement}
	\newtheorem{algorithm}[theorem]{Algorithm}
	\newtheorem{axiom}[theorem]{Axiom}
	\newtheorem{case}[theorem]{Case}
	\newtheorem{claim}[theorem]{Claim}
	\newtheorem{conclusion}[theorem]{Conclusion}
	\newtheorem{condition}[theorem]{Condition}
	\newtheorem{conjecture}[theorem]{Conjecture}
	\newtheorem{corollary}[theorem]{Corollary}
	\newtheorem{criterion}[theorem]{Criterion}
	\newtheorem{definition}{Definition} % Number definitions on their own
	\newtheorem{derivation}{Derivation} % Number derivations on their own
	\newtheorem{example}[theorem]{Example}
	\newtheorem{exercise}[theorem]{Exercise}
	\newtheorem{lemma}[theorem]{Lemma}
	\newtheorem{notation}[theorem]{Notation}
	\newtheorem{problem}[theorem]{Problem}
	\newtheorem{proposition}{Proposition} % Number propositions on their own
	\newtheorem{remark}[theorem]{Remark}
	\newtheorem{solution}[theorem]{Solution}
	\newtheorem{summary}[theorem]{Summary}
	%\numberwithin{equation}{section}
	\newcommand\ve{\varepsilon}
	\graphicspath{{./figures/}}
	\renewcommand\theenumi{\roman{enumi}}
	\DeclareMathOperator*{\Max}{Max}
	\bibliographystyle{aer}

\begin{document}



\begin{spacing}{1.5}

%section 1

\LARGE{Fitting Hazard Rates with Polynomials} \normalsize

We have data on mortality hazard rates from age 1 to age $90$.

We will fit these to the following univariate polynomial function.

\begin{equation}
	\ln \rho_{is} = \sum_{j=1}^J \beta_{is} \left(\frac{s}{90}\right)^j
\end{equation}
where $i$ indexes the country and $s$ is the age.  We fit each country separately in different regressions.

The vector of beta coefficients, $B_i$, can be estimated using OLS

\begin{equation}
	B_i = (X'X)^{-1}X'Y_i
\end{equation}
where $X$ is an $S\times(J+1)$ matrix of ages raised to various powers, and $Y_i$ is a $(J+1)\times 1$ vector of the natural log of mortality hazard rates for country $i$.

We fit these polynomials and save only the regression coefficients, the $B_i$s, to pass to the program.

In our Python program, to generate mortality hazard rates for agents that live for $S$ periods we use the regression equation above replacing 90 with $S$.  Note this gives us the one-year hazard rate at various age intervals.  To adjust for changes in the length of the period we must do further adjustments as shown below.

\begin{equation}
	\rho_{is} = 1 - \left(1-e^{\left[\sum_{j=1}^J \beta_{is} \left(\frac{s}{S}\right)^j\right]}\right)^{\frac{70}{S}}
\end{equation}


\end{spacing}

\newpage
\bibliography{Bibliography}
\end{document}